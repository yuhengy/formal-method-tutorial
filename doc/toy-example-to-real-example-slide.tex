\documentclass{beamer}


%%%%%%%%%%%%%%%%%%%%%%%%%
%%%% PREAMBLES BEGIN %%%%
%%%%%%%%%%%%%%%%%%%%%%%%%

%\usetheme{Berlin}

\AtBeginSection[]
{
  \begin{frame}
    \frametitle{Table of Contents}
    \tableofcontents[currentsection]
  \end{frame}
}

\AtBeginSubsection[]
{
  \begin{frame}
    \frametitle{Table of Contents}
    \tableofcontents[currentsubsection]
  \end{frame}
}


%%%%%%%%%%%%%%%%%%%%%%%
%%%% PREAMBLES END %%%%
%%%%%%%%%%%%%%%%%%%%%%%






\title{Formal Method on Hardware Security}
\author{}
\institute{}
\date{Oct 2022}

\begin{document}

\frame{\titlepage}








%%%%%%%%%%%%%%%%%%%%%%%%
%%%% SECTIONS BEGIN %%%%
%%%%%%%%%%%%%%%%%%%%%%%%

\begin{frame}
\frametitle{Table of Contents}
\tableofcontents
\end{frame}




\section{CPU Models}

\begin{frame}
\frametitle{CPU Models}
\begin{itemize}
  \Large
  \item<1-> RTL Implementation
  \item<2-> ISA Specification\onslide<3->{: A set of implementations}
\end{itemize}
\end{frame}




\section{Refinement Property}

\begin{frame}
\frametitle{Refinement Property}
\begin{itemize}
  \item<1-> RTL refines ISA \onslide<9->{{\color{red} regarding to commit cycle of each instruction. (How to define the commit cycle of a OoO CPU?)}}
  \item<2-> 1-cycle ALU refines symbolic-cycle ALU \onslide<8->{{\color{red} regarding to RTL implementations.}}
  \item<3-> 1-entry FIFO refines 2-entry FIFO \onslide<7->{{\color{red} regarding to enq(), deq() event sequences.}} \onslide<11->{{\color{red} Bluespec cares about event, Verilog cares about signals}}
  \alt<1-6>{\begin{itemize}
    \item<4-6> compare the enqueue(), dequeue() trace of the FIFO
    \item<5-6> 1-entry: enq(), deq(), enq(), deq() ...
    \item<5-6> 2-entry: enq(), deq() ... or enq(), enq(), deq(), ... or ...
    \item<6> Not correct in Verilog, is a good property in Bluespec
  \end{itemize}}
  \item \onslide<12-> The real implementation refines our model in the paper regarding to ... :)
\end{itemize}
\end{frame}




\section{More CPU Models}

\begin{frame}
\frametitle{More CPU Models}
\begin{itemize}
  \item<1-> RTL Implementation
  \item<2-> RTL, but replace ALU with a symbolic-cycle ALU
  \item<2-> RTL, but replace all caches with symbolic-cyle MEM
  \item<1-> ISA Specification: A set of implementations
\end{itemize}
\end{frame}



\section{More Properties on CPU}

\begin{frame}
\frametitle{More Properties on CPU}
\begin{itemize}
  \item<1-> Example: Memory Safety: any memory read/write instruction should within the boundary
  \item<2-> Safety property and Liveness property
  \begin{itemize}
    \item<3-> Safety property: something bad will not happen
    \item<3-> Liveness property: something good will eventually happen (no live lock)
  \end{itemize}
  \item<4-> Single-trace property and Hyperproperty
\end{itemize}
\end{frame}



\subsection{Single-trace Property and Hyperpropert}

\begin{frame}
\frametitle{Single-trace Property and Hyperproperty}
\begin{itemize}
  \item<1-> Memory Safety: any memory read/write instruction should within the boundary
  \item<2-> Non-interference property: Say we have a combinational function $y_1, y_2 = f(x_1, x_2)$
  \begin{align*}
    & \text{$x_1$ not interfere $y_1$} ~~ \text{iff} \\
    \forall ~~& x_1, x_1', x_2, \\
    \text{let} ~~& y_1, y_2 = f(x_1, x_2), \\
    ~~~~~~& y_1', y_2' = f(x_1', x_2), \\
    \text{we have}, ~~& y_1 = y_1'
  \end{align*}
  \item<3-> Hyperproperty is a property on a multiple traces
\end{itemize}
\end{frame}



\subsection{Example: Non-interference Property}

\begin{frame}
\frametitle{Example: Non-interference Property}
\begin{itemize}
  \item<1-> A property on a pair of execution traces: for a function $y_1, y_2 = f(x_1, x_2)$:
  \begin{align*}
    & \text{$x_1$ not interfere $y_1$} ~~ \text{iff} \\
    \forall ~~& x_1, x_1', x_2, \\
    \text{let} ~~& y_1, y_2 = f(x_1, x_2), \\
    ~~~~~~& y_1', y_2' = f(x_1', x_2), \\
    \text{we have}, ~~& y_1 = y_1'
  \end{align*}
  \item<2-> Verify it with single trace: taint analysis: for a function $y_1, y_2 = f(x_1, x_2)$, we design an augment $yt_1, yt_2 = ft(x_1, xt_1, x_2, xt_2)$, such that:
  \begin{align*}
    & \text{$x_1$ not interfere $y_1$} ~~ \text{iff} \\
    \forall ~~& x_1, x_2, \\
    \text{let} ~~& yt_1, yt_2 = ft(x_1, xt_1, x_2, xt_2), \\
    \text{we have}, ~~& yt_1 = 0
  \end{align*}
\end{itemize}
\end{frame}



\subsection{An Estimation Scheme: Taint Analysis}

\begin{frame}
\frametitle{An Estimation Scheme: Taint Analysis}
\begin{itemize}
  \item<1-> Taint analysis: for a function $y_1, y_2 = f(x_1, x_2)$, we design an augment $yt_1, yt_2 = ft(x_1, xt_1, x_2, xt_2)$, such that:
  \begin{align*}
    & \text{$x_1$ not interfere $y_1$} ~~ \text{iff} \\
    \forall ~~& x_1, x_2, \\
    \text{let} ~~& yt_1, yt_2 = ft(x_1, xt_1, x_2, xt_2), \\
    \text{we have}, ~~& yt_1 = 0
  \end{align*}
  \item<2-> Finally, let me show a tool to automatically augment verilog with Taint functions. \onslide<3->{{\color{red} More tools (JasperGold and $\sim$100 more are coming \url{http://mtlcad.mit.edu/setup.html}).}}
\end{itemize}
\end{frame}




\begin{frame}
\frametitle{Taint Analysis: Gate-Level Information Flow Tracking (GLIFT)}
\begin{itemize}
  \item<1-> Augment at each gate
  \item<2-> When compose gates together or compose cycles together, will be conservative.
  \item<1-> More on \url{https://ieeexplore.ieee.org/document/5948366}
\end{itemize}
\end{frame}




%%%%%%%%%%%%%%%%%%%%%%
%%%% SECTIONS END %%%%
%%%%%%%%%%%%%%%%%%%%%%

\end{document}

