\documentclass{article}


%%%%%%%%%%%%%%%%%%%%%%%%%
%%%% PREAMBLES BEGIN %%%%
%%%%%%%%%%%%%%%%%%%%%%%%%
\usepackage[lmargin=1.5in, lmargin=1.5in, tmargin=0.5in, bmargin=0.5in]{geometry} % page size

\usepackage[utf8]{inputenc}
\usepackage{amsmath} % for align
\usepackage{hyperref} % for url




%%%% Title Formats BEGIN %%%%
\renewcommand\thesubsection{\thesection.\alph{subsection}}
%%%% Title Formats END %%%%




%%%% Comments BEGIN %%%%
\usepackage[dvipsnames]{xcolor}

\newif\ifdraft % set to false to remove comments and todos
\drafttrue
%\draftfalse

\newcommand{\yuheng}[1]{\ifdraft{{\footnotesize\color{purple}[Yuheng: #1]}}\fi}
%%%% Comments END %%%%

%%%%%%%%%%%%%%%%%%%%%%%
%%%% PREAMBLES END %%%%
%%%%%%%%%%%%%%%%%%%%%%%








\title{Formal Method on Hardware Security}
\author{}
\date{September 2022}

\begin{document}

\maketitle








%%%%%%%%%%%%%%%%%%%%%%%%
%%%% SECTIONS BEGIN %%%%
%%%%%%%%%%%%%%%%%%%%%%%%


\section{A Note on Induction}


\paragraph{A Mistake from Last Week.}
Last week, we have a state machine and want to verify a property hold for each state of the machine.
Formally,
\begin{align*}
& \forall ~~ \text{input} ~~ in_0, in_1, \dots, in_n, \\
& \text{Let} ~~ S_0 \xrightarrow[]{in_0} S_{1} \xrightarrow[]{in_1} \cdots \xrightarrow[]{in_n} S_{n+1} \\
& \text{We have} ~~ PropertyHold(S_0), PropertyHold(S_1), \dots, PropertyHold(S_n)
\end{align*}

The property I verify for the induction step is:
\begin{align*}
& \forall ~~ \text{state $S_i$, input $in_i$, $in_{i+1}$}, \\
& \text{Let} ~~ S_i \xrightarrow[]{in_i} S_{i+1} \xrightarrow[]{in_{i+1}} S_{i+2} \\
& \text{We have} ~~ PropertyHold(S_{i+1}) \Rightarrow PropertyHold(S_{i+2})
\end{align*}

It is a little weird since why not just use:
\begin{align*}
& \forall ~~ \text{state $S_i$, input $in_i$}, \\
& \text{Let} ~~ S_i \xrightarrow[]{in_i} S_{i+1} \\
& \text{We have} ~~ PropertyHold(S_{i}) \Rightarrow PropertyHold(S_{i+1})
\end{align*}

The answer is that yes, this bottom version definitely looks more natural for our toy example.
The reason that I mistakenly use the above version is that I was used to the way I use induction on my CPU verification, which uses the above version.
Now, let me explain why the above version is the one I use in my CPU, and eventually, you will realize induction can be more powerful than you think.




\paragraph{A State Machine with Output.}

In real design, you probably want a state machine to have output so that we can express our security property more easily.
For example, your CPU may send something out through IO at each cycle, like print out some data on the screen.
Formally, we denote a simulation step as:
\begin{align*}
S_i \xrightarrow[out_i]{in_i} S_{i+1}
\end{align*}

And your security property may look like this output should not contain secret.
Formally,
\begin{align*}
& \forall ~~ \text{input} ~~ in_0, in_1, \dots, in_n, \\
& \text{Let} ~~ S_0 \xrightarrow[out_0]{in_0} S_{1} \xrightarrow[out_1]{in_1} \cdots \xrightarrow[out_n]{in_n} S_{n+1} \\
& \text{We have} ~~ PropertyHold(out_0), PropertyHold(out_1), \dots, PropertyHold(out_n)
\end{align*}

Now, if you compare this property with the above one, the only difference is whether the property is for state or output.
And when the property is for output, it can be a little difficult to think about the induction step of verification.
The key is that, to represent $out_i$ and $out_{i+1}$, we need three states.
Formally, the property for the induction step is:
\begin{align*}
& \forall ~~ \text{state $S_i$, input $in_i$, $in_{i+1}$}, \\
& \text{Let} ~~ S_i \xrightarrow[out_{i}]{in_i} S_{i+1} \xrightarrow[out_{i+1}]{in_{i+1}} S_{i+2} \\
& \text{We have} ~~ PropertyHold(out_{i}) \Rightarrow PropertyHold(out_{i+1})
\end{align*}

\textbf{We need to define an arbitrary $out_i$ for induction. However, we need to define an arbitrary pre-states $S_i$ to represent the arbitrary $out_i$.}





%%%%%%%%%%%%%%%%%%%%%%
%%%% SECTIONS END %%%%
%%%%%%%%%%%%%%%%%%%%%%

\end{document}
