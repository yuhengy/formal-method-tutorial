\documentclass{article}


%%%%%%%%%%%%%%%%%%%%%%%%%
%%%% PREAMBLES BEGIN %%%%
%%%%%%%%%%%%%%%%%%%%%%%%%
\usepackage[lmargin=1.5in, lmargin=1.5in, tmargin=0.5in, bmargin=0.5in]{geometry} % page size

\usepackage[utf8]{inputenc}
\usepackage{amsmath} % for align
\usepackage{hyperref} % for url




%%%% Title Formats BEGIN %%%%
\renewcommand\thesubsection{\thesection.\alph{subsection}}
%%%% Title Formats END %%%%




%%%% Comments BEGIN %%%%
\usepackage[dvipsnames]{xcolor}

\newif\ifdraft % set to false to remove comments and todos
\drafttrue
%\draftfalse

\newcommand{\yuheng}[1]{\ifdraft{{\footnotesize\color{purple}[Yuheng: #1]}}\fi}
%%%% Comments END %%%%

%%%%%%%%%%%%%%%%%%%%%%%
%%%% PREAMBLES END %%%%
%%%%%%%%%%%%%%%%%%%%%%%








\title{Formal Method on Hardware Security}
\author{}
\date{September 2022}

\begin{document}

\maketitle








%%%%%%%%%%%%%%%%%%%%%%%%
%%%% SECTIONS BEGIN %%%%
%%%%%%%%%%%%%%%%%%%%%%%%


\section{A Note on Induction}


\paragraph{A Mistake from Last Week.}
Last week, we have a state machine and want to verify a property hold for each state of the machine.
Formally,
\begin{align*}
& \forall ~~ \text{input} ~~ in_0, in_1, \dots, in_n, \\
& \text{Let} ~~ S_0 \xrightarrow[]{in_0} S_{1} \xrightarrow[]{in_1} \cdots \xrightarrow[]{in_n} S_{n+1} \\
& \text{We have} ~~ PropertyHold(S_0), PropertyHold(S_1), \dots, PropertyHold(S_n)
\end{align*}

The property I verify for the induction step is:
\begin{align*}
& \forall ~~ \text{state $S_i$, input $in_i$, $in_{i+1}$}, \\
& \text{Let} ~~ S_i \xrightarrow[]{in_i} S_{i+1} \xrightarrow[]{in_{i+1}} S_{i+2} \\
& \text{We have} ~~ PropertyHold(S_{i+1}) \Rightarrow PropertyHold(S_{i+2})
\end{align*}

It is a little weird since why not just use:
\begin{align*}
& \forall ~~ \text{state $S_i$, input $in_i$}, \\
& \text{Let} ~~ S_i \xrightarrow[]{in_i} S_{i+1} \\
& \text{We have} ~~ PropertyHold(S_{i}) \Rightarrow PropertyHold(S_{i+1})
\end{align*}




\paragraph{A State Machine with Output.}

And your security property is for each output.
Formally,
\begin{align*}
& \forall ~~ \text{input} ~~ in_0, in_1, \dots, in_n, \\
& \text{Let} ~~ S_0 \xrightarrow[out_0]{in_0} S_{1} \xrightarrow[out_1]{in_1} \cdots \xrightarrow[out_n]{in_n} S_{n+1} \\
& \text{We have} ~~ PropertyHold(out_0), PropertyHold(out_1), \dots, PropertyHold(out_n)
\end{align*}

Now, formally, the property for the induction step is:
\begin{align*}
& \forall ~~ \text{state $S_i$, input $in_i$, $in_{i+1}$}, \\
& \text{Let} ~~ S_i \xrightarrow[out_{i}]{in_i} S_{i+1} \xrightarrow[out_{i+1}]{in_{i+1}} S_{i+2} \\
& \text{We have} ~~ PropertyHold(out_{i}) \Rightarrow PropertyHold(out_{i+1})
\end{align*}

\textbf{We need to define an arbitrary $out_i$ for induction. However, we need to define an arbitrary pre-states $S_i$ to represent the arbitrary $out_i$.}





%%%%%%%%%%%%%%%%%%%%%%
%%%% SECTIONS END %%%%
%%%%%%%%%%%%%%%%%%%%%%

\end{document}
